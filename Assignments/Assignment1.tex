\documentclass[10pt]{article}
\usepackage[dvips]{graphics}
\usepackage{graphicx}
\usepackage{setspace}
\usepackage{xcolor}
\usepackage{tabu}
\usepackage{makecell}
\usepackage{multirow}
\usepackage{subfigure}
\setlength{\textwidth}{6.27in} \setlength{\textheight}{9.5in}
\setlength{\oddsidemargin}{-0.25in}
\setlength{\evensidemargin}{-0.25in} \setlength{\topmargin}{-0.25in}
\setlength{\parindent}{0pt} \setlength{\parskip}{0mm}
\newtheorem{theorem}{\bf Theorem}
\newenvironment{proof}{\pf}{\endpf}
\def\pn{\par\noindent}
\usepackage{amsmath}
\usepackage{amssymb}
\usepackage{float, epsfig, floatflt, here}
\usepackage{cite}
\usepackage{longtable}
\usepackage{epstopdf}
\usepackage{pdfpages}
\begin{document}
\graphicspath{ {figures/} }
\onehalfspacing
\setlength{\unitlength}{5mm} \baselineskip 0.6cm
\setcounter{page}{2} \pagenumbering{arabic}
\begin{center}
\Large{\textbf{\underline{Assignment 1}}}
\vskip 100pt
\Large{\textbf{}}
\vskip 50pt
\large{\textit{\textbf{Name: Devosmita Chatterjee}}}\\
\large{\textit{\textbf{Email Id: chatterjeedevosmita267@gmail.com}}}
\end{center}
\newpage
\title{\bf{{}}}
\date{}%\overline{}
\maketitle

\section*{{\large{1\hskip 20pt Introduction}}}
The cancer dataset consists of values for breast cancer mortality from 1950 to 1960 and the adult white female population in 1960 for 301 counties in North Carolina, South Carolina, and Georgia.

a. Make a histogram of the population values for cancer mortality.

b. What are the population mean and total cancer mortality? What are the population variance and standard deviation?

c. Simulate the sampling distribution of the mean of a sample of 25 observations
of cancer mortality.

d. Draw a simple random sample of size 25 and use it to estimate the mean and
total cancer mortality.

e. Estimate the population variance and standard deviation from the sample of
part (d).

f. Form 95% confidence intervals for the population mean and total from the
sample of part (d). Do the intervals cover the population values?

g. Repeat parts (d) through (f) for a sample of size 100.

l. Stratify the counties into four strata by population size. Randomly sample six
observations from each stratum and form estimates of the population mean
and total mortality.

m. Stratify the counties into four strata by population size. What are the sampling fractions for proportional allocation and optimal allocation? Compare
the variances of the estimates of the population mean obtained using simple
random sampling, proportional allocation, and optimal allocation.

n. How much better than those in part (m) will the estimates of the population
mean be if 8, 16, 32, or 64 strata are used instead?
\section*{{\large{3\hskip 20pt Conclusion}}}

\footnotesize
\begin{thebibliography}{99}

\bibitem{nc}










\end{thebibliography}

\end{document}













